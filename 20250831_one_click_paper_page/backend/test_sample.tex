\documentclass{article}
\usepackage{amsmath}
\usepackage{amsfonts}
\usepackage{amssymb}

\title{Sample Academic Paper}
\author{Test Author}
\date{\today}

\begin{document}

\maketitle

\begin{abstract}
This is a sample academic paper to test the conversion pipeline. It includes mathematical formulas, sections, and basic formatting to verify that our conversion tools work correctly.
\end{abstract}

\section{Introduction}

This paper demonstrates the conversion capabilities of our academic paper to website system. The system supports multiple input formats including LaTeX, PDF, and DOCX files.

\section{Mathematical Content}

Here are some mathematical expressions to test formula rendering:

Inline math: $E = mc^2$

Display math:
\begin{equation}
\int_{-\infty}^{\infty} e^{-x^2} dx = \sqrt{\pi}
\end{equation}

Matrix example:
\begin{equation}
A = \begin{pmatrix}
a & b \\
c & d
\end{pmatrix}
\end{equation}

\section{Lists and Structure}

\subsection{Ordered List}
\begin{enumerate}
\item First item
\item Second item
\item Third item
\end{enumerate}

\subsection{Unordered List}
\begin{itemize}
\item Bullet point one
\item Bullet point two
\item Bullet point three
\end{itemize}

\section{Conclusion}

This sample document tests the basic functionality of our conversion pipeline. It should be successfully converted to HTML while preserving the mathematical content and document structure.

\end{document}
